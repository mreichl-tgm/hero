\documentclass[11pt]{article}

\usepackage[utf8]{inputenc}
\usepackage[T1]{fontenc}
\usepackage[default]{raleway}
\usepackage[english]{babel}
\usepackage[dvipsnames]{xcolor}
\usepackage[marginal, norule, perpage]{footmisc}

\renewcommand{\thefootnote}{\Roman{footnote}}

\usepackage{hyperref}
\hypersetup{
  colorlinks=true,
  linkcolor=MidnightBlue,
  urlcolor=MidnightBlue
}

\title{
  \textbf{The Hero Project}\\
  \large{Feasibility Study}
  \linebreak
  \linebreak
  \small{\texttt{01001000\\01000101\\01010010\\01001111}}
}

\author{
  \begin{tabular}{rl}
    \textbf{Team:}
    & \textsc{Djordjevic} Filip\\
    & \textsc{Reichl} Markus \small{\textit{(Project Manager}}\\
    & \textsc{Tekin} Abdurrahim\\
    & \textsc{Wellner} Florian\\
    \\
    \textbf{Supervision:}
    & \textsc{Dolezal} Dominik
  \end{tabular}
}

\begin{document}
\begin{titlepage}
  \clearpage
  \maketitle
  \thispagestyle{empty}
  
  \begin{abstract}
    \begin{flushleft}
      The Hero Project is a \textbf{bullethell roguelike multiplayer RPG} providing a very unique storytelling and gameplay experience.
      
      Its main goal as a school project is to create a solid base for a video game ready for further development. The core should be highly abstract, stable and easy to extend.
      \linebreak
      \linebreak
      This document goes into detail with the projects technical feasibility and reviews different approaches resulting into a clear conclusion and the best way to go.
     \end{flushleft}
  \end{abstract}
\end{titlepage}

\tableofcontents
\newpage

\section{Introduction}
The Hero Project is a \textbf{
  bullethell
  \footnote{A video game sub-genre where the screen is usually covered in bullets.}
  roguelike
  \footnote{A video game sub-genre which, based on the \href{http://roguebasin.com/roguelike-definition}{Roguebasin Interpretation}, is defined by 
    \begin{itemize}
      \item \textbf{Permanent Failure:} The player is encouraged to take responsibility for the risks he takes.
      \item \textbf{Procedural Environments:} Most of the game world is generated and provides complexity in resources and other elements of the game.
      \item \textbf{Resources:} The player can manage a limited amount of resources.
    \end{itemize}
    The \href{http://roguebasin.com/roguelike-definition}{Roguebasin}, \href{http://roguebasin.com/index.php?title=Berlin_Interpretation}{Berlin} and \href{http://roguetemple.com/roguelike-definition}{Roguetemple} Interpretation may give a more detailed explaination on this subject.
  }
  multiplayer
  \footnote{A multiplayer game allows but does not require clients to play together. The Hero Project is not meant to be played online only!}
  RPG\footnote{A roleplay game is a game in which the player assumes the roles of characters in a fictional setting and takes responsibility for his acting either thorugh literal acting.}
} providing a very unique storytelling and gameplay experience.

It is highly inspired by previous titles in the roguelike and roleplay game genres, the most noteworthy being \href{https://realmofthemadgod.com}{Realm of the Mad God}, \href{http://www.devolverdigital.com/games/view/titan-souls}{Titan Souls} and \href{http://dodgeroll.com/gungeon/}{Enter the Gungeon}.

While all mentioned games are using a topdown pixelart setting The Hero Project is drawn in a very unique combination of high and low resolutions.

\section{Project Goals}
The main goal of The Hero Project as a school project is to create a solid base for a video game ready for further development.
This base includes all elements listed in the \textit{Functional Requirements} and \textit{Technical Requirements} sections found in the \textit{Requirement Specification} document.
Because The Hero Project as a school project only has to provide the core of the game there is no need to review the economic or legal feasibility.

\newpage

\section{Technical Feasibility}
With a video game being an enormous software project this study is mostly technical which requires the reader to have at least basic knowledge about video games and software development.
Technical terms will still be explained as footnotes at the bottom of each page.

\subsection{Team}
Because the majority of the product will be developed by a rather small team each members technical experience and skill has a huge influence on the projects schedule and quality.
The Hero Projects core as a school project is developed by a static team of 4 members and 1 professor listed below.
\paragraph{\textsc{Djordjevic} Filip} ~\\
\begin{tabular}{ll}
\texttt{Roles} & Sounddesigner\\
\texttt{Areas of expertise} & Music and Sounddesign
\end{tabular}
\paragraph{\textsc{Reichl} Markus} ~\\
\begin{tabular}{ll}
\texttt{Roles} & Project Manager \textit{\small{Product Owner)}}\\
\texttt{Areas of expertise} & Concept Art and System Architecture
\end{tabular}
\paragraph{\textsc{Tekin} Abdurrahim} ~\\
\begin{tabular}{ll}
\texttt{Roles} & Artist\\
\texttt{Areas of expertise} & Art and Design
\end{tabular}
\paragraph{\textsc{Wellner} Florian} ~\\
\begin{tabular}{ll}
\texttt{Roles} & Software Developer\\
\texttt{Areas of expertise} & Software Development
\end{tabular}
\paragraph{\textsc{Dolezal} Dominik} ~\\
\begin{tabular}{ll}
\texttt{Roles} & Professor \textit{\small{(Supervisor)}}\\
\texttt{Areas of expertise} & Software Development
\end{tabular}
While providing a wide variety of skills the majority of the team is also well attuned. Everyone has at least some experience with video games but only a few have already worked on one.

\end{document}
