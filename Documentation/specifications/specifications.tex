\documentclass[11pt]{article}

\usepackage[utf8]{inputenc}
\usepackage[T1]{fontenc}
\usepackage[default]{raleway}
\usepackage[english]{babel}
\usepackage[dvipsnames]{xcolor}
\usepackage[marginal, norule, perpage]{footmisc}
\usepackage{microtype}

\renewcommand{\thefootnote}{\Roman{footnote}}

\usepackage{hyperref}
\hypersetup{
  colorlinks=true,
  linkcolor=MidnightBlue,
  urlcolor=MidnightBlue
}

\title{
  \textbf{The Hero Project}\\
  \large{Functional Specification}
  \linebreak
  \linebreak
  \small{\texttt{01001000\\01000101\\01010010\\01001111}}
}

\author{
  \begin{tabular}{rl}
    \textbf{Team:}
    & \textsc{Djordjevic} Filip\\
    & \textsc{Reichl} Markus \small{\textit{(Project Manager)}}\\
    & \textsc{Tekin} Abdurrahim\\
    & \textsc{Wellner} Florian\\
    \\
    \textbf{Supervision:}
    & \textsc{Dolezal} Dominik
  \end{tabular}
}

\begin{document}
\begin{titlepage}
  \clearpage
  \maketitle
  \thispagestyle{empty}
  
  \begin{abstract}
    \begin{flushleft}
      The Hero Project is a \textbf{bullethell roguelike multiplayer RPG} providing a very unique storytelling and gameplay experience.
      
      Its main goal as a school project is to create a solid base for a video game ready for further development. The core should be highly abstract, stable and easy to extend.
      \linebreak
      \linebreak
      This document defines all features the product has to provide, as a response to the \textit{Requirement Specification} document, using the results of the teams \textit{Feasibility Study}.
     \end{flushleft}
  \end{abstract}
\end{titlepage}

\tableofcontents
\newpage

\section{Introduction}
The Hero Project is a \textbf{
  bullethell
  \footnote{A video game sub-genre where the screen is usually covered in bullets.}
  roguelike
  \footnote{A video game sub-genre which, based on the \href{http://roguebasin.com/roguelike-definition}{Roguebasin Interpretation}, is defined by 
    \begin{itemize}
      \item \textbf{Permanent Failure:} The player is encouraged to take responsibility for the risks he takes.
      \item \textbf{Procedural Environments:} Most of the game world is generated and provides complexity in resources and other elements of the game.
      \item \textbf{Resources:} The player can manage a limited amount of resources.
    \end{itemize}
    The \href{http://roguebasin.com/roguelike-definition}{Roguebasin}, \href{http://roguebasin.com/index.php?title=Berlin_Interpretation}{Berlin} and \href{http://roguetemple.com/roguelike-definition}{Roguetemple} Interpretation may give a more detailed explanation on this subject.
  }
  multiplayer
  \footnote{A multiplayer game allows but does not require clients to play together. The Hero Project is not meant to be played online only!}
  RPG\footnote{A roleplay game is a game in which the player assumes the roles of characters in a fictional setting and takes responsibility for his acting either through literal acting.}
} providing a very unique storytelling and gameplay experience.

It is highly inspired by previous titles in the roguelike and roleplay game genres, the most noteworthy being \href{https://realmofthemadgod.com}{Realm of the Mad God}, \href{http://www.devolverdigital.com/games/view/titan-souls}{Titan Souls} and \href{http://dodgeroll.com/gungeon/}{Enter the Gungeon}.

While all mentioned games are using a topdown pixelart setting, the Hero Project is drawn in a very unique combination of high and low resolutions.

\section{Project Goals}
The main goal of The Hero Project as a school project is to create a solid base for a video game ready for further development.
This base includes all elements listed in the \hyperref[sec:fs]{Functional} and \hyperref[sec:ts]{Technical} Specifications sections found later in this document.

\subsection{Targets}
\paragraph{Targets}
The team is going to provide a core game as mentioned earlier and also a prototype for presentation.
\paragraph{Non Targets}
The team is not thought to provide a demo version or even a complete game.

\newpage

\section{Functional Specifications}\label{sec:fs}
\subsection{\texttt{/FS10/} Artificial Intelligence}\label{subsec:fs10ai}
Entities\footnote{Every interacting object in the game is referred to as a (game) entity} are similar to a \hyperref[subsec:fs20character]{character} and controlled by the game can interact with their environment by using \hyperref[subsec:fs30effects]{effects}.
The game itself will differentiate between two main types of AIs, NPCs\footnote{A non player character is a character which is not controlled by the player} and Enemies, for most of the time. 
This does not imply that an AI is either a NPC or an enemy, they just differentiate through their behaviour.
\paragraph{NPCs}
NPCs may interact with the player through conversation or providing resources.
\paragraph{Enemies}
While NPCs are most likely to be friendly enemies will use their effects to harm the \hyperref[subsec:fs60player]{player}.
They can also be used to provide \hyperref[subsec:fs41items]{items} and other \hyperref[subsec:fs40resources]{resources} for the player, when an enemy is killed by a attack controlled by the player.
\paragraph{Other AIs}
AIs different from NPCs and enemies could be objects provided by the environment which interact with the player by using effects.
\subsection{\texttt{/FS20/} Character}\label{subsec:fs20character}
Characters are entities which are directly controlled by the client or the game. Every character has the ability to \hyperref[subsec:fs201movement]{move}, \hyperref[subsec:fs202levelling]{level}, use \hyperref[subsec:fs30effects]{effects} and relies on \hyperref[subsec:fs50attributes]{attributes}.
\subsubsection{\texttt{/FS20.1/} Movement}\label{subsec:fs201movement}
The characters movement can be controlled by the game or the client using an input of his input of choice.
\subsubsection{\texttt{/FS20.2/} Levelling}\label{subsec:fs202levelling}
A \hyperref[subsec:fs20character]{character} is able to gain levels which directly modify its \hyperref[subsec:fs50attributes]{attributes}.
Attributes can also influence AIs, mostly by modifying their attributes, spawn rates\footnote{Entities created in a limited amount of time} or resources.
Levels can be given by default or increase on specific actions. Common ways for the player to improve his level includes fighting enemies or completing quests\footnote{A search or pursuit made in order to find or obtain something. In this case a quest describes a task for the player to complete.}.
\subsection{\texttt{/FS30/} Effects}\label{subsec:fs30effects}
Entities can be given effects which allow them to interact with their environment.
They can be activated, enabled or interact permanently. Common effects include \hyperref[subsec:fs50attributes]{attribute} manipulation, conversations and also attacks.
\hyperref[subsec:fs60player]{Players} should be able the control their effects using an input of their choice.
\subsubsection{\texttt{/FS30.1/} Attack}
A specific effect which is also implemented in the core game is the attack. Attacks are used to reduce another entities \textit{life} attribute which leads to its death.
Attacks interact by firing \hyperref[par:projectiles]{projectiles}.
\paragraph{Projectiles}\label{par:projectiles}
Projectiles are used by attacks to harm other entities and also to display it. Short range attacks are implemented as a single projectile with a bigger hitbox\footnote{An area in which one object interacts with another}.
\subsection{\texttt{/FS40/} Resources}\label{subsec:fs40resources}
Resources are all collectable objects in the game. They do not interact themselves but can be interacted with by a character through his \hyperref[subsec:fs42inventory]{inventory}.
\subsection{\texttt{/FS41/} Items}\label{subsec:fs41items}
Items are resources that can be given \hyperref[subsec:fs30effects]{effects}, which can only be activated by another entity. 
They should also be ready to be displayed to the player although this feature will not be implemented in the core game.
\subsection{\texttt{/FS42/} Inventory}\label{subsec:fs42inventory}
An inventory keeps a limited amount of items and can hold base items, equipment and other resources.
\subsection{\texttt{/FR50/} Attributes}\label{subsec:fs50attributes}
All entities using this functionality are influenced by 8 attributes called \textit{health, mana, damage, defence, vitality, wisdom, speed} and \textit{agility}. 
Their values are used as variables in the entities actions and can also be modified. The number of attributes can also be increased during development.
Attributes in the core game have to be accessible but not yet interactable for an AI.
\subsection{\texttt{/FR60/} Player}\label{subsec:fs60player}
A single client is able to control at least one characters actions and manage its \hyperref[subsec:fs40resources]{resources}.

\section{Usability Specifications}\label{sec:us}
\subsection{\texttt{/US10/} Minimalist User Interface}
While providing a huge amount of complexity the game also provides clarity for the user using a minimalist user interface. This improves the usability a lot and makes the game more enjoyable for the player.

\section{Technical Specifications}\label{sec:ts}
\subsection{\texttt{/TS10/} Cross-Platform}
The game runs on multiple platforms including all major desktop operating systems and popular consoles using the cross-platform features of the Unity game engine.
\subsection{\texttt{/TS20/} Extendable}
The core game will be highly extendable through abstraction, decoupling and maintainability.
\subsection{\texttt{/TS30/} Multiplayer Ready}
Because the actual game is thought to be played on a server with multiple clients the core is developed ready for multiple clients.
A real multiplayer is not necessary in the core game but definitely nice to have.

\section{Quality}
Extendability and functionality are high priority for the core game because of their huge influence on the later game.
High quality graphics are only required for presentation and are therefore considered low priority.

\end{document}